\documentclass[12pt,a4paper]{report}

% --- Paquetes base (Overleaf/pdfLaTeX friendly) ---
\usepackage[spanish, es-nodecimaldot]{babel}
\usepackage[utf8]{inputenc}
\usepackage[T1]{fontenc}
\usepackage[a4paper,margin=2.5cm]{geometry}
\usepackage{hyperref}
\usepackage{titlesec}
\usepackage{parskip}
\usepackage{graphicx}
\usepackage{fancyhdr}
\usepackage{tabularx}
\usepackage{longtable}
\usepackage{enumitem}
\usepackage{xcolor}
\usepackage{listings}

% --- Hipervínculos bonitos ---
\hypersetup{
  colorlinks=true,
  linkcolor=black,
  urlcolor=blue,
  citecolor=black
}

% --- Encabezados y pies de página ---
\pagestyle{fancy}
\fancyhf{}
\lhead{TDF HQ — Manual de Usuario (v1)}
\rhead{\leftmark}
\cfoot{\thepage}

% --- Estilo de secciones ---
\titleformat{\section}{\normalfont\Large\bfseries}{\thesection.}{0.6em}{}
\titleformat{\subsection}{\normalfont\large\bfseries}{\thesubsection.}{0.5em}{}
\titleformat{\subsubsection}{\normalfont\normalsize\bfseries}{\thesubsubsection.}{0.4em}{}

% --- Listados de código (bash, JSON) ---
\lstdefinestyle{tdfcode}{
  basicstyle=\ttfamily\small,
  frame=single,
  framerule=0.4pt,
  framesep=6pt,
  rulecolor=\color{black!30},
  xleftmargin=0pt,
  xrightmargin=0pt,
  breaklines=true,
  columns=fullflexible,
  keepspaces=true,
  showstringspaces=false
}
\lstdefinelanguage{bash}{
  morekeywords={curl,cd,ls,grep,sed,awk,printf,echo,export,source,stack,createdb,psql},
  sensitive=true
}
\lstdefinelanguage{json}{
  morestring=[b]",
  morecomment=[l]{//},
  literate=
   *{0}{{{\color{black}0}}}{1}
    {1}{{{\color{black}1}}}{1}
    {2}{{{\color{black}2}}}{1}
    {3}{{{\color{black}3}}}{1}
    {4}{{{\color{black}4}}}{1}
    {5}{{{\color{black}5}}}{1}
    {6}{{{\color{black}6}}}{1}
    {7}{{{\color{black}7}}}{1}
    {8}{{{\color{black}8}}}{1}
    {9}{{{\color{black}9}}}{1}
}

% --- Portada ---
\title{\vspace{-2cm}\textbf{Manual de Usuario}\\[2mm]\large TDF HQ — CRM, Agendamiento, Paquetes e Invoicing (v1)}
\author{TDF Records / TDF Estudio}
\date{\today}

\begin{document}
\maketitle
\tableofcontents
\newpage

\section{Introducción y alcance}
\textbf{TDF HQ} centraliza las operaciones del estudio y sello:
\begin{itemize}[topsep=2pt]
  \item \textbf{CRM:} artistas, clientes, profesores, ingenieros.
  \item \textbf{Agendamiento:} reservas básicas (sesiones, ensayos, clases).
  \item \textbf{Paquetes de clases:} productos y compras.
  \item \textbf{Facturación:} creación de facturas en borrador.
\end{itemize}

\noindent
Esta versión (v1) expone un \textbf{backend API} en Haskell (Servant + Persistent + PostgreSQL) con migraciones automáticas y datos semilla para pruebas.
\medskip

\noindent
\textit{No incluido aún:} autenticación/roles en API, detección de conflictos de recursos, deducción automática de créditos por asistencia, PDF de factura y SRI, Google Calendar. El modelo de datos ya lo contempla para siguientes versiones.

\section{Requisitos del sistema}
\begin{itemize}[topsep=2pt]
  \item \textbf{PostgreSQL} 14–16 (recomendado 16).
  \item \textbf{Haskell Stack} (Stack descarga GHC automáticamente).
  \item macOS o Linux (Windows funciona con WSL/instalador).
  \item Conexión a internet para la primera compilación.
\end{itemize}

\section{Instalación rápida}
\subsection{macOS (Homebrew)}
\begin{lstlisting}[style=tdfcode,language=bash]
brew install haskell-stack postgresql@16
brew services start postgresql@16
echo 'export PATH="/opt/homebrew/opt/postgresql@16/bin:$PATH"' >> ~/.zshrc
source ~/.zshrc
\end{lstlisting}

\subsection{Ubuntu/Debian}
\begin{lstlisting}[style=tdfcode,language=bash]
sudo apt update
sudo apt install haskell-stack postgresql postgresql-contrib
sudo service postgresql start
\end{lstlisting}

\section{Configuración}
\subsection{Variables de entorno}
Desde la carpeta del proyecto:
\begin{lstlisting}[style=tdfcode,language=bash]
set -a; source config/default.env; set +a
\end{lstlisting}

\noindent
Valores por defecto:
\begin{lstlisting}[style=tdfcode,language=bash]
DB_HOST=127.0.0.1
DB_PORT=5432
DB_USER=postgres
DB_PASS=postgres
DB_NAME=tdf_hq
APP_PORT=8080
\end{lstlisting}

\subsection{Base de datos}
\begin{lstlisting}[style=tdfcode,language=bash]
createdb tdf_hq || psql -d postgres -c "CREATE DATABASE tdf_hq;"
psql -d postgres -c "ALTER USER postgres WITH PASSWORD 'postgres';"
\end{lstlisting}

\section{Compilación y ejecución}
\begin{lstlisting}[style=tdfcode,language=bash]
stack build
stack run
\end{lstlisting}

\noindent
Salida esperada:
\begin{lstlisting}[style=tdfcode,language=bash]
Running DB migrations...
Starting server on port 8080
\end{lstlisting}

\noindent
Prueba de salud:
\begin{lstlisting}[style=tdfcode,language=bash]
curl http://localhost:8080/health
# {"status":"ok","db":"ok"}
\end{lstlisting}

\section{Datos semilla (desarrollo)}
\textbf{Sólo} para entorno de desarrollo (crea artistas, profesores, salas, paquete Guitar 24h y equipo inicial):
\begin{lstlisting}[style=tdfcode,language=bash]
curl -X POST http://localhost:8080/admin/seed
\end{lstlisting}

\section{Uso por áreas}
\subsection{CRM (Party/Personas)}
\paragraph{Crear persona}
\begin{lstlisting}[style=tdfcode,language=bash]
curl -X POST http://localhost:8080/parties \
  -H "Content-Type: application/json" \
  -d '{"cDisplayName":"Juano Ledesma","cIsOrg":false,
       "cPrimaryEmail":"juano@tdf.com"}'
\end{lstlisting}

\paragraph{Listar personas}
\begin{lstlisting}[style=tdfcode,language=bash]
curl http://localhost:8080/parties
\end{lstlisting}

\paragraph{Consultar/Actualizar}
\begin{lstlisting}[style=tdfcode,language=bash]
curl http://localhost:8080/parties/1
curl -X PUT http://localhost:8080/parties/1 \
  -H "Content-Type: application/json" \
  -d '{"uInstagram":"@juano"}'
\end{lstlisting}

\paragraph{Asignar rol} (valores: Artist, Teacher, Engineer, Customer, etc.)
\begin{lstlisting}[style=tdfcode,language=bash]
curl -X POST http://localhost:8080/parties/1/roles \
  -H "Content-Type: application/json" \
  -d '"Artist"'
\end{lstlisting}

\subsection{Agendamiento (Bookings)}
\paragraph{Crear reserva}
\begin{lstlisting}[style=tdfcode,language=bash]
curl -X POST http://localhost:8080/bookings \
  -H "Content-Type: application/json" \
  -d '{"cbTitle":"Rehearsal Arkabuz",
       "cbStartsAt":"2025-10-01T19:00:00Z",
       "cbEndsAt":"2025-10-01T21:00:00Z",
       "cbStatus":"Confirmed"}'
\end{lstlisting}

\paragraph{Listar reservas}
\begin{lstlisting}[style=tdfcode,language=bash]
curl http://localhost:8080/bookings
\end{lstlisting}

\subsection{Paquetes de clases}
\paragraph{Ver productos}
\begin{lstlisting}[style=tdfcode,language=bash]
curl http://localhost:8080/packages/products
\end{lstlisting}

\paragraph{Registrar compra}
\begin{lstlisting}[style=tdfcode,language=bash]
curl -X POST http://localhost:8080/packages/purchases \
  -H "Content-Type: application/json" \
  -d '{"buyerId":1,"productId":1}'
\end{lstlisting}

\subsection{Facturación}
\paragraph{Crear factura (borrador)}
\begin{lstlisting}[style=tdfcode,language=bash]
curl -X POST http://localhost:8080/invoices \
  -H "Content-Type: application/json" \
  -d '{"ciCustomerId":1,"ciSubtotalCents":150000,
       "ciTaxCents":18000,"ciTotalCents":168000}'
\end{lstlisting}

\paragraph{Listar facturas}
\begin{lstlisting}[style=tdfcode,language=bash]
curl http://localhost:8080/invoices
\end{lstlisting}

\section{Referencia rápida de endpoints}
\begin{longtable}{p{4.2cm}p{7.6cm}p{3.2cm}}
\hline
\textbf{Endpoint} & \textbf{Descripción} & \textbf{Método} \\
\hline
\endhead
/health & Salud del servidor & GET \\
/parties & Lista/creación de personas (Party) & GET, POST \\
/parties/:id & Obtener/actualizar una persona & GET, PUT \\
/parties/:id/roles & Asignar un rol a la persona & POST \\
/bookings & Listar/crear reservas & GET, POST \\
/packages/products & Ver productos de paquete & GET \\
/packages/purchases & Registrar compra de paquete & POST \\
/invoices & Listar/crear facturas (borrador) & GET, POST \\
/admin/seed & Semilla de datos (dev) & POST \\
\hline
\end{longtable}

\section{Roles y permisos recomendados (cuando se active Auth)}
\begin{itemize}[topsep=2pt]
  \item \textbf{Admin/Manager:} acceso total.
  \item \textbf{Reception:} CRM básico, ventas de paquetes, reservas.
  \item \textbf{Teacher/Engineer:} ver agenda propia, actualizar asistencia (próximo).
  \item \textbf{Accounting:} facturas, pagos, notas de crédito, reportes.
\end{itemize}

\section{Respaldo y restauración (DB)}
\paragraph{Backup}
\begin{lstlisting}[style=tdfcode,language=bash]
pg_dump -h 127.0.0.1 -U postgres -d tdf_hq -F c \
  -f tdf_hq_$(date +%F).dump
\end{lstlisting}

\paragraph{Restore}
\begin{lstlisting}[style=tdfcode,language=bash]
pg_restore -h 127.0.0.1 -U postgres -d tdf_hq \
  --clean --create tdf_hq_YYYY-MM-DD.dump
\end{lstlisting}

\section{Solución de problemas}
\begin{itemize}[topsep=2pt]
  \item \textbf{\texttt{stack: command not found}}: instalar Stack y añadir \texttt{\$HOME/.local/bin} al PATH.
  \item \textbf{\texttt{createdb: command not found}}: instalar Postgres y añadir \texttt{bin} al PATH.
  \item \textbf{Plan de Stack no resuelto (\texttt{persistent-postgresql})}:
  relajar versión en \texttt{.cabal} a \texttt{$\ge$2.13} \textit{o} añadir \texttt{extra-deps} sugeridos en \texttt{stack.yaml}.
  \item \textbf{Error \texttt{libpq: invalid connection option "pool"}}:
  remover \texttt{pool=10} del connection string; el pool lo maneja Haskell.
  \item \textbf{Warning \texttt{libpq.dylib built for newer macOS}}: es sólo advertencia.
\end{itemize}

\section{Roadmap sugerido}
\begin{enumerate}[topsep=2pt]
  \item Autenticación Google OAuth + RBAC.
  \item Conflictos de recursos (salas/profesores/equipo) + buffers.
  \item Ledger de paquetes (descuento al marcar asistencia).
  \item Integración Google Calendar (dos vías).
  \item Facturas PDF con marca TDF + numeración (SRI).
  \item Inventario: check-in/out con QR y mantenimiento.
  \item Reportes: revenue por servicio, utilización, aging, expiración de paquetes.
\end{enumerate}

\section{Glosario}
\begin{description}[style=nextline]
  \item[Party] Persona/organización con roles (Artist, Teacher, Engineer, Customer, Vendor).
  \item[Booking] Reserva en calendario (ensayo, sesión, clase).
  \item[PackageProduct / Purchase] Producto de horas/clases y su compra.
  \item[Invoice] Factura con totales (líneas/PDF en próximas versiones).
  \item[Resource] Recurso físico o humano (sala, profesor, ingeniero, equipo).
\end{description}

\appendix
\section*{Apéndice A — Ejemplos listos para pegar}
\addcontentsline{toc}{section}{Apéndice A — Ejemplos listos para pegar}

\paragraph{Crear cliente}
\begin{lstlisting}[style=tdfcode,language=bash]
curl -X POST http://localhost:8080/parties \
  -H "Content-Type: application/json" \
  -d '{"cDisplayName":"El Bloque (Manager)","cIsOrg":true,
       "cPrimaryEmail":"manager@elbloque.com"}'
\end{lstlisting}

\paragraph{Asignar rol Customer}
\begin{lstlisting}[style=tdfcode,language=bash]
curl -X POST http://localhost:8080/parties/2/roles \
  -H "Content-Type: application/json" \
  -d '"Customer"'
\end{lstlisting}

\paragraph{Booking de grabación}
\begin{lstlisting}[style=tdfcode,language=bash]
curl -X POST http://localhost:8080/bookings \
  -H "Content-Type: application/json" \
  -d '{"cbTitle":"Recording - Live Room",
       "cbStartsAt":"2025-10-02T15:00:00Z",
       "cbEndsAt":"2025-10-02T18:00:00Z",
       "cbStatus":"Confirmed"}'
\end{lstlisting}

\paragraph{Vender paquete Guitar 24h}
\begin{lstlisting}[style=tdfcode,language=bash]
curl http://localhost:8080/packages/products
curl -X POST http://localhost:8080/packages/purchases \
  -H "Content-Type: application/json" \
  -d '{"buyerId":1,"productId":1}'
\end{lstlisting}

\paragraph{Crear factura}
\begin{lstlisting}[style=tdfcode,language=bash]
curl -X POST http://localhost:8080/invoices \
  -H "Content-Type: application/json" \
  -d '{"ciCustomerId":1,"ciSubtotalCents":50000,
       "ciTaxCents":6000,"ciTotalCents":56000}'
\end{lstlisting}

\vfill
\begin{center}
\footnotesize{\textit{TDF HQ v1 — Este manual describe la versión base del backend. Las funcionalidades avanzadas se agregarán de forma incremental.}}
\end{center}

\end{document}
