% !TEX TS-program = xelatex
\documentclass[11pt,a4paper]{scrartcl}

% --- Encoding & fonts ---
\usepackage{fontspec}
% Fonts (pick widely-available options for Overleaf + local builds)
\setmainfont{DejaVu Serif}
\setsansfont{DejaVu Sans}
\setmonofont{DejaVu Sans Mono}

% --- Layout ---
\usepackage[a4paper,margin=24mm]{geometry}
\usepackage{setspace}
\setstretch{1.10}

% --- Color & graphics ---
\usepackage{xcolor}
\definecolor{claroRed}{HTML}{DA291C}
\definecolor{tdfBlack}{HTML}{111111}
\definecolor{tdfGray}{HTML}{6F6F6F}
\definecolor{softGray}{HTML}{F4F4F6}

\usepackage{graphicx}
\usepackage{tikz}
\usetikzlibrary{calc}
\usepackage{hyperref}
\hypersetup{
  colorlinks=true,
  linkcolor=claroRed,
  urlcolor=claroRed,
  citecolor=claroRed
}

% --- Tables & lists ---
\usepackage{tabularx}
\usepackage{enumitem}

% --- Header / footer ---
\usepackage{fancyhdr}
\setlength{\headheight}{34pt} % avoid fancyhdr warning
\pagestyle{fancy}
\fancyhf{}
\renewcommand{\headrulewidth}{0.4pt}
\renewcommand{\footrulewidth}{0pt}
\fancyfoot[C]{\small\color{tdfGray} \thepage}

% --- Section styling ---
\usepackage{titlesec}
\titleformat{\section}{\Large\bfseries\color{tdfBlack}}{}{0em}{}
\titleformat{\subsection}{\large\bfseries\color{tdfBlack}}{}{0em}{}
\titlespacing*{\section}{0pt}{1.25em}{0.6em}
\titlespacing*{\subsection}{0pt}{1.0em}{0.45em}

% --- Helpers ---
\newcommand{\SafeIncludeGraphics}[2][]{%
  \IfFileExists{#2}{\includegraphics[#1]{#2}}{%
    \begin{tikzpicture}
      \draw[tdfGray] (0,0) rectangle (6,3);
      \node[align=center,tdfGray,font=\sffamily\footnotesize] at (3,1.5)
      {Imagen no encontrada\\\texttt{#2}};
    \end{tikzpicture}%
  }%
}

% --- Assets (keep these names as-is in Overleaf) ---
\newcommand{\HeroImage}{assets/hero.jpg}
% NOTE: tdf_logo.png contains large transparent padding, which makes it appear
% much smaller than Claro in the header even at the same height.
% We therefore use a cropped version for headers.
\newcommand{\LogoTDF}{assets/tdf_logo_header.png}
\newcommand{\LogoClaro}{assets/claro_logo.png}
\newcommand{\GlyphTDFWhite}{assets/tdf_glyph_white.png}

\newcommand{\PhotoPortrait}{assets/portrait.jpg}
\newcommand{\PhotoPortraitTwo}{assets/portrait2.jpg}
\newcommand{\PhotoKeys}{assets/keys2.jpg}
\newcommand{\PhotoKeysAlt}{assets/keys.jpg}
\newcommand{\PhotoSax}{assets/sax.jpg}
\newcommand{\PhotoPads}{assets/pads.jpg}
\newcommand{\PhotoGuitar}{assets/guitar.jpg}
\newcommand{\PhotoWide}{assets/wide1.jpg}
\newcommand{\PhotoSelfie}{assets/selfie.jpg}

% Uniform image helper for the reference grid
\newcommand{\GridPhoto}[1]{\SafeIncludeGraphics[width=\linewidth,height=42mm,keepaspectratio]{#1}}

% Header logos
\fancyhead[L]{\SafeIncludeGraphics[height=10mm]{\LogoTDF}}
\fancyhead[R]{\SafeIncludeGraphics[height=10mm]{\LogoClaro}}

\begin{document}

% =========================
% Cover
% =========================
\begin{titlepage}
\thispagestyle{empty}

\begin{tikzpicture}[remember picture,overlay]
  % Full-bleed hero
  % Full-bleed cover: fill by width and let the page crop the overflow.
  \node[anchor=center] at (current page.center)
    {\SafeIncludeGraphics[width=\paperwidth]{\HeroImage}};

  % Subtle “swap” watermark on the drum head (keeps the spirit, without hurting readability)
  \node[opacity=0.14,rotate=-10] at ($(current page.center)+(0mm,-18mm)$)
    {\SafeIncludeGraphics[width=95mm]{\GlyphTDFWhite}};

  % Clear glyph on the darkest zone for legibility
  \node[anchor=north west,xshift=18mm,yshift=-18mm] at (current page.north west)
    {\SafeIncludeGraphics[width=42mm]{\GlyphTDFWhite}};

  % Title block
  \node[anchor=south east,xshift=-18mm,yshift=18mm] at (current page.south east) {
    \begin{minipage}{0.62\paperwidth}
      \raggedleft
      \color{white}
      {\Huge\bfseries Propuesta}\par
      {\LARGE\bfseries Contenido y Producción Musical}\par
      \vspace{5mm}
      {\large \textbf{TDF Records} \;\textbullet\; \textbf{Claro}}\par
      \vspace{4mm}
      {\small \today}
    \end{minipage}
  };
\end{tikzpicture}

\vfill
\end{titlepage}

\pagenumbering{arabic}

% =========================
% Executive summary
% =========================
\section*{Resumen}
\addcontentsline{toc}{section}{Resumen}

\noindent
Esta propuesta presenta un formato de contenido musical grabado en vivo (audio y video) con estética cinematográfica, pensado para amplificar la presencia de \textbf{Claro} dentro de una escena cultural activa, auténtica y altamente compartible.\par

\vspace{2mm}
\noindent
\textbf{TDF Records} se encarga de la producción integral: preproducción, grabación multicámara, captura de audio profesional, edición, mezcla/master y entregables listos para publicación.

% =========================
\section{Objetivo}
\begin{itemize}[leftmargin=*,itemsep=4pt]
  \item Asociar a \textbf{Claro} con creatividad, cultura y talento local mediante piezas audiovisuales de alta calidad.
  \item Crear un formato replicable (por temporadas) para generar comunidad y consistencia en redes.
  \item Entregar contenido con estética premium, optimizado para \textbf{YouTube} y recortes para \textbf{Instagram / TikTok}. 
\end{itemize}

% =========================
\section{Formato propuesto}
\subsection{Sesiones en vivo}
\begin{itemize}[leftmargin=*,itemsep=4pt]
  \item 1 artista o banda por episodio.
  \item 1 a 2 canciones grabadas en vivo.
  \item Audio multipista + mezcla/master final.
  \item Video multicámara (look cinematográfico) + edición.
\end{itemize}

\subsection{Microcontenidos}
\begin{itemize}[leftmargin=*,itemsep=4pt]
  \item 6 a 12 clips verticales por episodio (15--45s).
  \item Teasers, hooks, behind-the-scenes, momentos musicales.
  \item Entrega pensada para pauta y orgánico.
\end{itemize}

% =========================
\section{Producción}
\subsection{Preproducción}
\begin{itemize}[leftmargin=*,itemsep=4pt]
  \item Curaduría de artistas (con aprobación de Claro).
  \item Scouting visual (locación / set / arte).
  \item Plan de rodaje + guion técnico (cámaras, luces, audio).
\end{itemize}

\subsection{Rodaje y grabación}
\begin{itemize}[leftmargin=*,itemsep=4pt]
  \item 4 a 6 cámaras (según episodio y complejidad).
  \item Captura de audio profesional (mics dedicados + multipista).
  \item Dirección de fotografía e iluminación.
\end{itemize}

\subsection{Postproducción}
\begin{itemize}[leftmargin=*,itemsep=4pt]
  \item Edición de video (corte multicámara, color, gráficos).
  \item Mezcla y master del audio final.
  \item Entrega de masters + versiones optimizadas para redes.
\end{itemize}

% =========================
\section{Recursos técnicos}
\begin{tabularx}{\linewidth}{@{}X X@{}}
\textbf{Audio} & \textbf{Video} \\
\hline
\vspace{1mm}
\begin{itemize}[leftmargin=*,itemsep=3pt]
  \item Micrófonos premium (dinámicos / condensador)
  \item Grabación multipista (DAW)
  \item Monitoreo y control de señal
  \item Mezcla y master en estudio
\end{itemize}
&
\vspace{1mm}
\begin{itemize}[leftmargin=*,itemsep=3pt]
  \item Cámaras cine / mirrorless 4K
  \item Iluminación LED (look cinematográfico)
  \item Dirección de fotografía
  \item Edición + color grading
\end{itemize}
\\
\end{tabularx}

% =========================
\section{Cronograma estimado (por episodio)}
\begin{itemize}[leftmargin=*,itemsep=4pt]
  \item \textbf{Preproducción:} 3--5 días
  \item \textbf{Rodaje:} 1 día
  \item \textbf{Edición video + mezcla/master:} 7--10 días
  \item \textbf{Entrega final:} 10--15 días desde rodaje
\end{itemize}

% =========================
\section{Entregables}
\begin{itemize}[leftmargin=*,itemsep=4pt]
  \item 1 video principal por episodio (YouTube, 4K / 1080p).
  \item 6--12 microclips verticales (IG Reels / TikTok).
  \item Audio master (WAV) + versión streaming (MP3).
  \item Mini pack de fotos (frames seleccionados) para posts.
\end{itemize}

% =========================
\section{Enlaces a episodios}
\noindent
Por limitaciones de acceso desde algunos entornos, aquí dejamos el enlace de la playlist y el primer episodio. Si deseas que dejemos \textit{todos} los links individuales (uno por video), envíanos la lista de URLs o los IDs, y los insertamos en 1 minuto.\par

\vspace{2mm}
\begin{itemize}[leftmargin=*,itemsep=4pt]
  \item Playlist: \href{https://www.youtube.com/watch?v=9387ent0ELc\&list=PLORPSiW9rnkjSYKaBSAX-QqoVf_9b29EP}{Abrir en YouTube}
  \item Episodio 1: \href{https://www.youtube.com/watch?v=9387ent0ELc}{Ver video}
\end{itemize}

% =========================
\section{Visual de referencia}
\noindent
Algunas imágenes (estilo, energía y ejecución) para comunicar el look del proyecto:\par

\vspace{2mm}
\setlength{\tabcolsep}{3pt}
\renewcommand{\arraystretch}{1}
\begin{tabular}{@{}p{0.32\linewidth}p{0.32\linewidth}p{0.32\linewidth}@{}}
\GridPhoto{\PhotoPortrait} &
\GridPhoto{\PhotoPortraitTwo} &
\GridPhoto{\PhotoKeys} \\[3pt]
\GridPhoto{\PhotoSax} &
\GridPhoto{\PhotoPads} &
\GridPhoto{\PhotoGuitar} \\[3pt]
\GridPhoto{\PhotoKeysAlt} &
\GridPhoto{\PhotoWide} &
\GridPhoto{\PhotoSelfie} \\
\end{tabular}

\vspace{6mm}
\noindent\textbf{Contacto:} \href{mailto:diego@tdfrecords.net}{diego@tdfrecords.net}\;\textbullet\; Quito, Ecuador

\end{document}
